\documentclass[12pt]{article}
\usepackage[margin=3cm,paper=a4paper]{geometry}
\usepackage{lmodern}
\usepackage[T1]{fontenc}
\usepackage[latin1]{inputenc}
\usepackage[brazil]{babel}
\usepackage{indentfirst}

\author{Nelson Lu�s Dias}
\date{\today}
\title{Minha segunda aula, 2020-05-13}

\begin{document}

\maketitle

\section{Introdu��o}\label{sec:intro}

Um �timo exemplo. Eu posso comentar coisas no arquivo-fonte .tex
Isso � um coment�rio em LaTeX. Devemos fazer coment�rios gen�ricos que n�o
tenham nenhum sentido ruim, bl� bl� bl�, isso s� tem o objetivo 
de tornar o primeiro par�grafo um pouco mais longo para que todos o 
identifiquem como
um par�grafo, e n�o como uma linha solta.

% Eu posso comentar coisas no arquivo-fonte .tex
% Isso � um coment�rio em LaTeX

O ciclo de desenvolvimento envolve entrar no editor,
fazer modifica��es, salvar o arquivo, recompilar em LaTeX,
e revisualizar no visualizador de pdf.

Mas agora, como � que eu posso dizer que fiz 100\% do trabalho.

\section{Fontes de material did�tico}

Na se��o \ref{sec:intro}, n�s demos uma breve introdu��o.

Um bom texto se chama ``The not so short introduction to \LaTeX''.

\# cerquinhas agora devem funcionar

Se eu quiser comprar um m�quina de cortar cabela, ela $\backslash$ 
vai custar cerca de R\$ 500 reais. O que mais? \^{}, \&, \_, \{, \}, \~{},

Vamos come�ar um novo par�grafo e escrever que houve uma epidemia
de gripe espanhola nos anos 1918--19. Ou: as rugosidades do fundo
do canal variam no interval 0.001--0.01 m. Isso n�o � o mesmo
que 1918-1919.

A noite em S�o Petersburgo 
--- uma das cidades mais setentrionais do planeta --- pode 
durar muito mais do que 12 horas.

Em S�o Petersburgo, as temperaturas v�o de $-20$ at� $+30$ graus cent�grados.

Em S�o Petersburgo, as temperaturas v�o de -20 at� +30 graus cent�grados.

Como sabemos, $0 \ge -1$. ou: $a < b$ ou: $a \le b$.









\end{document}