\documentclass[biolinum,graylinen,20pt]{marmot}
%
% Font shape `T1/ccr/b/n' undefined -> encontrei concmath.sty
% ccr10, etc. not found ... --> ccfonts
% ditto eorm10.mf -> ecc
% ditto ccr10.tfm -> concrete
%
\usepackage[T1]{fontenc}
\usepackage[latin1]{inputenc}
\usepackage[round]{natbib}  
%\usepackage[brazil]{babel} % muda o idioma para o Portugu�s  
\usepackage{comment}
\usepackage{fontawesome}
\usepackage{amssymb}
\usepackage{mhchem}
\usepackage{mathrsfs}
\usepackage{mathtools}
\usepackage{booktabs}
\usepackage{multirow}
\usepackage{soul}
\usepackage{upgreek}


\input{math.tex}
\input{ira.tex}

\renewcommand{\emph}[1]{\textbf{\textit{#1}}}
%\parskip=6pt



%\leftlogo{lemmaufprlogo.pdf}
\centername{\color{SteelBlue} XI Workshop Brasileiro de Micrometeorologia}

\leftlogo{ufpr-new-t.png}
\rightlogo{wkshop-logo-t.png}


\title{Scalar flux measurement: progress and challenges.}
\author{Nelson Lu�s Dias}
\affil{\normalsize Department of Environmental Engineering, Federal University of Paran�,
  Brazil\\
  \faEnvelopeO: \url{nldias@ufpr.br}  \hspace*{2cm} \faChain: \url{www.nldias.github.io}}
\date{XI Brazilian Micrometeorology Workshop, 20-22 Nov 2019}



\begin{document}

\maketitle

\noslide[Acknowledgements]

This talk draws from collaborative research with may people, including:

\vspace*{1cm}

Andr� Luiz Diniz,
Bianca Luhm Crivellaro,
Cl�o-Quaresma Dias-Jr,
Diana M. Cancelli,
Dornelles Vissotto Jr.,
Einara Zahn,
Fernando Augusto S. Armani,
Lucas Emilio Hoeltgebaum,
L�via Freire Souza Grion,
Marcelo Chamecki,
Rodrigo Branco Rodakoviski,
Tom�s Lu�s G. Chor,

and several others.

\noslide[Thanks]

{\large for the invitation:}

\vfill

{\large It is an honor.}

\vfill

\mcontents

%\nosection{Acknowledgements}


\msection{Motivation}

\mslide[My approach in this talk]

{\begin{itemize}\large
  \item Have a quick look at the fundamentals of scalar flux measurement.
  \item Discuss some (relatively recent) progress that has been made.
  \item \ul{Discuss difficulties and challenges.}
  \item Point to interesting research issues that deserve our attention.
\end{itemize}}



\mslide[Why is it important to measure scalar fluxes?]

\begin{minipage}[t]{0.51\textwidth}
  \vspace*{0pt}

  Scalar fluxes are everywhere in Earth Science:
\begin{itemize}\tightitems
  \item The ever-important heat fluxes $H$ and $LE$
  \item Fluxes of greenhouse gases: \ce{CO2}, \ce{CH4}, \ce{N2O}, \ldots
  \item All sorts of pollutant fluxes
  \item VOC fluxes, \ce{O3} fluxes, \ldots (towards SOAs and cloud condensation nuclei)
\end{itemize}
    {\small Right: figure from \cite{fuentes2016linking}}

\end{minipage}
\hfill
\begin{minipage}[t]{0.48\textwidth}
  \vspace*{0pt}
  \centering
  \includegraphics[width=0.8\textwidth]{interactions}
\end{minipage}

%%{\footnotesize      Below: figure from \url{https://www.power-technology.com/news/smog-reduction-technologies-save-eu-citizens-e183bn-2025/attachment/smoke-air-pollution-air-sun-smog-sunset-pollution/}}
\vspace*{1cm}


\begin{minipage}[t]{0.48\textwidth}
  \vspace*{0pt}\centering
  \includegraphics[width=0.7\textwidth]{smog.jpg}
\end{minipage}
\hfill
\begin{minipage}[t]{0.48\textwidth}
  \vspace*{0pt}
Their accurate estimation, therefore, impacts all sorts of areas of human
activity:
\begin{itemize}\tightitems
  \item Weather prediction
  \item Climate simulation
  \item Air pollution modelling
  \item Hydrology and streamflow forecasting
\end{itemize}
\end{minipage}

\mslide[Why is it difficult to measure scalar fluxes?]

\begin{itemize}\tightitems
  \item The fluxes are turbulent
  \item Depending on the scalar, sensor response time and accuracy can be unable to resolve turbulent fluctuations
  \item They occur under non-ideal conditions:
  \begin{itemize}\tightitems
    \item in nonstationary periods
    \item under local advective conditions
    \item over complex terrain or topography
  \end{itemize}
  \item Reynolds's analogy may not apply $\Rightarrow$ \emph{Monin-Obukhov
    Similarity Theory (MOST) may apply to scalar $a$, but not to $b$.}
\end{itemize}\pause

{\large What lies ahead?}

\begin{minipage}{0.5\textwidth}
  \vspace*{0pt}
  \includegraphics[width=0.9\textwidth]{tothebeach.jpg}
\end{minipage}
\begin{minipage}{0.5\textwidth}
  \vspace*{0pt}
\begin{itemize}\tightitems
  \item There is still a lot to be learned about them
  \item Research on fluxes is not only useful,
  \item It is a fun (but long) walk (as always).
\end{itemize}
\end{minipage}

\msection{Fundamentals}

\mslide[Definition of terms]

\begin{itemize}\tightitems
  \item $\rho$ ($\mathsf{M\,L^{-3}}$) is fluid density; $\vut{u}=(u,v,w)$ is fluid velocity.
  \item $\chi$ is a scalar.
  \item $\rho_\chi$ is scalar density
   or \emph{volumetric
    concentration}; $\dimof{\rho_\chi}=\mathsf{M}_\chi\mathsf{L}^{-3}$.
   \item $c_\chi=\rho_\chi/\rho$ is scalar mass concentration: $\dimof{c_\chi} =
   \mathsf{M}_\chi\mathsf{M}^{-1}$.
   \item $r_\chi = \rho_\chi/\rho_d$ is scalar mixing ratio.
   \item $\mathbf{j}_\chi$ is the (molecular) diffusive flux of $\chi$.
\end{itemize}

\textbf{Lower concentrations are harder to measure}:

\begin{center}
{\begin{tabular}{rcrr}
\toprule
Gas        & $\chi=$ &\% & $\rho_\chi\, (\mathrm{g\,m^{-3}})$ \\
\midrule
\ce{N2}    & \multirow{2}{*}{$\left. \phantom{\bigg|} \right\} s$} & 78.000000  & 908.3679 \\
\ce{O2}    &                    &21.000000  & 279.3485 \\
\ce{H2O}   & $v$                  & 2.500000  &  18.7229 \\
\ce{CO2}   & $c$ & 0.040000     &   0.7318 \\
\ce{CH4}   & $m$ & 0.000175     &   0.0012 \\
\ce{N2O}   & $n$ & 0.000031     &   0.0006 \\
\bottomrule
\end{tabular}
}
\end{center}


\mslide[Many assumptions and difficult measurements]


  \includegraphics[width=0.58\textwidth]{caixa}\hfill
  \includegraphics[width=0.40\textwidth]{adv}

Control volume balances:
\begin{equation*}
   -\oint_{\mathscr{S}} (\vut{n}\bdot\vut{j}_\chi)\,\md{S} 
   = \parder{}{t}\int_{\mathscr{V}} \rho_\chi \,\md{V} + \oint_{\mathscr{S}}
   \rho_\chi(\vut{n}\bdot\vut{u})\,\md{S} \; \Rightarrow 
   F_\chi = \int_0^{z_a} \parder{\aavg{\rho_\chi}}{t} \,\md{z} +
   \aavg{w\rho_\chi}\Big|_{z_a}
\end{equation*}

Above, $\aavg{w\rho_\chi}$ is an ensemble average at height $z_a$
(the measurement height). \emph{Eddy covariance} measurements produce \emph{estimates}
$\bavg{w\rho_\chi}$ of $\aavg{w\rho_\chi}$.

\vfill

Above, right: figure from
\cite{leuning.zegelin.ea--measurement.horizontal.vertical.advection}. 

\vfill

\mslide[But there is a lot of complexity about $\overline{w\rho_\chi}$:]

(Will not discuss the first three)
\begin{itemize}
  {\color{SlateBlue}\item Advection is complicated (no standard procedure except for tons of
  equipments). For a really nice study, see
  \cite{leuning.zegelin.ea--measurement.horizontal.vertical.advection}.}
  {\color{SlateBlue}\item The storage term $\int_0^{z_a} \parder{\bavg{\rho_\chi}}{t} \,\md{z}$ is
  doable, but requires vertical profiles of $\bavg{\rho_\chi}$ measurements
  (calibrated, intercompared, \ldots)}
  {\color{SlateBlue}\item WPL correction is mandatory: $\overline{w\rho_\chi} =
  \overline{w}\,\overline{\rho_\chi} + \overline{w'\rho_\chi'}$.}
  \item Random \emph{measurement} errors are large, and probably larger than
  previously thought.
  \item \emph{Modeling} errors related to incomplete knowledge of the physics
  are also large.
  \item Monin-Obukhov Similarity Theory may not apply equally to all scalars.
  \item Corrections for sensor response and geometry may be needed.
  \item Flow distortion is likely a larger problem than previously thought, and
  can affect the value of the measured scalar fluxes.
  \item Calibration is difficult and tricky.
\end{itemize}



\msection{Errors}

\mslide[Different types of errors]

  \textbf{Random errors:}
  We measure an estimate (usually a time average) of an
  ensemble average.

Analogous with

\begin{minipage}[t]{0.4\textwidth}
  \begin{align*}
    \bavg{x} &= \frac{1}{n}\sum_{i=1}^n x_i, \\
    \Var\{ \bavg{x} \} &= \frac{\Var\{x\}}{n}.
  \end{align*}
\end{minipage}
\begin{minipage}[t]{0.6\textwidth}
    \begin{align}
      \bavg{x}_\Delta &= \frac{1}{\Delta}\int_0^\Delta x(t)\,\md{t},\nonumber\\
      \MSE(\bavg{x})  &= \frac{2\mathscr{T}}{\Delta}\Var\{x\},
      \tag{\small Lumley-Panofsky}\\
      \MSE(\bavg{x})  &= \beta
      \Var\{x\}\left(\frac{\mathscr{T}_H}{\Delta}\right)^p.\tag{\small Relaxed Filtering Method}
    \end{align}
\end{minipage}

$\mathscr{T}$ is the integral scale \citep{lumley.panofsky--structure}; For
  $p<1$, \emph{it
    may not exist in the statistical sense}; this is an instance of the
  \emph{Hurst Phenomenon} in Turbulence \citep{dias2018hurst}.

  \pause
  \vfill

  \textbf{Modeling errors:}
We adopt a simplified set of assumptions to estimate
  the flux $F_\chi$, because \emph{we do not know all of the physics}. This
  introduces errors around the true values:\\[0.5cm] $\widehat{F_\chi} =
  f(\zeta,\bavg{u},\Delta \bavg{c_\chi}, \ldots)$, $\qquad$ but $\qquad\left|\widehat{F_\chi}-F_\chi\right|$
  can be large on average.

  \vfill

  \mslide[Random errors]

  Relaxed filtering method (RFM) \cite{dias2018hurst}:
  \stepwise{%
  \step{\begin{minipage}[t]{0.48\textwidth}
      \vspace*{0pt}%\small
    $\RMSE(\overline{w'c'}) = c\Delta^{-p/2}$
    \begin{itemize}\tightitems
       \item Similar to bootstrapping.
       \item Power-law fitting with turbulence data.
       \item $p$ and $\RMSE(\overline{w'c'})$ are calculated for each
         ``block'' of length $T$.
    \end{itemize}
    
    \includegraphics[width=\textwidth]{urmselaw_u2}
    \end{minipage}\hfill}
  \step{\begin{minipage}[t]{0.48\textwidth}
    \vspace*{0pt}%\small

    The RFM produces $\RMSE$ values \emph{considerably larger} (for scalar fluxes) than
    those from the Lumley-Panofsky equation:
    \begin{center}
      \includegraphics[width=0.7\textwidth]{5-escint_rmse_tijucas_wt_so.pdf}
    \end{center}
    Report $\RMSE$ estimates with the measurments!
  \end{minipage}}}


  
\mslide[Modeling errors]

\begin{itemize}\tightitems
\item MOST requires a very stringent set of assumptions: stationarity, no horizontal
advection, the inertial sublayer.
\item Even when these conditions are met, however, MOST may not apply.
\item Measurements made in the \emph{roughness sublayer} (where MOST \emph{does not} work) are
a serious obstacle.
\end{itemize}
\pause
\textbf{Indirect measurements in the roughness sublayer:}
\begin{itemize}\tightitems
  \item Direct (Eddy Covariance) flux measurements in the roughness sublayer are
restricted to a few scalars: temperature, $\mathrm{H_2O}$, $\mathrm{CO_2}$,
$\mathrm{CH_4}$.
  \item Some other scalars can also be measured by EC (ozone, $\mathrm{OH^-}$,
  isoprene, etc.), but at considerable difficulty (campaign-mode, etc.).
  \item $\Rightarrow$  need of \emph{indirect} methods based on slower scalar sensor
  measurements:
  \begin{enumerate}
    \item Flux-gradient (based on MOST).
    \item Relaxed eddy accumulation (also based on MOST).
  \end{enumerate}
\end{itemize}

\mslide[Examples from ATTO]


\stepwise{
\step{\begin{minipage}[t]{0.4\textwidth}
    \vspace*{0pt}
  REA (Relaxed Eddy Accumulation):
  \[
  \overline{w'c'} = b_c\sigma_w\left( \overline{c^+} - \overline{c^-}\right);
  \qquad b_c = \text{const.}
  \]
  \cite{zahn.dias.ea--scalarturbulentbehavior}: selection of small zenith angles
  ($|Z| \le 20^{\circ}$) recovers MOST-like behavior.
\end{minipage}}
\step{\begin{minipage}[t]{0.3\textwidth}
    \vspace*{0pt}
    {\centering All data.\\
  \includegraphics[width=\textwidth]{reatotal42c}}
\end{minipage}}
\step{\begin{minipage}[t]{0.3\textwidth}
  \vspace*{0pt}
  {\centering $|Z| \le 20^{\circ}$\\\includegraphics[width=\textwidth]{rea202042c}}
\end{minipage}}}
(But improvement is for a few hours during the day, and a few months per year)

\pause

\newcommand{\boxit}[1]{\parbox[t]{2.5cm}{\centering #1}}
{\begin{center}
Error statistics for flux-gradient estimates at the ATTO site. From:
    \cite{diasjr2019classical}.
\begin{tabular}{lcccccc}
\toprule
Flux & unit                                & \boxit{overall mean} & \boxit{absolute RMSE} & \boxit{relative RMSE} & BIAS     & $r$  \\
\midrule
$LE$ & $\mathrm{W\,m^{-2}}$                & 211.70       & 201.58  & 134\%     & $-$46.87 & 0.39 \\
$F$  & $\mathrm{\upmu{mol}\,m^{-2}\,s^{-1}}$ & $-$0.357     & 0.615   & 6521\%    & $+$0.033    & 0.16 \\
\bottomrule
\end{tabular}
\end{center}}

\mslide[\ldots And the inertial sublayer may not  exist over tall forests]


\begin{minipage}[t]{0.58\textwidth}
  \vspace*{0pt}

  \cite{diasjr2019classical} looked for the inertial sublayer at the ATTO site,
  but did not find it.

  \begin{center}
  \includegraphics[width=0.8\textwidth]{amazon-abl.jpg}
  \end{center}

  Without an inertial sublayer, the flux estimates will be poor if based on MOST.
\end{minipage}\hfill
\begin{minipage}[t]{0.38\textwidth}
  \vspace*{0pt}
  \includegraphics[width=\textwidth]{assku.png}
\end{minipage}

\mslide[Alas, sometimes MOST also fails in the inertial sublayer]

\begin{minipage}[t]{0.6\textwidth}
  \vspace*{0pt}
  \begin{minipage}[t]{0.59\textwidth}
    \vspace*{0pt}
    \includegraphics[width=\textwidth,height=0.25\textheight]{fufetch}\\
  \end{minipage}\hfill
  \begin{minipage}[t]{0.4\textwidth}
    \vspace*{0pt}\small
    On the right \cite{dias2017effect}
    \[
    \frac{\sigma_c}{c_*} \stackrel{?}{=} \phi_{c}(\zeta)
    \]
    for Furnas and Verde lakes
  \end{minipage}
  {Scalar flux number \cite{cancelli.dias.ea--dissimilarity.implicaitons}:
  $\displaystyle
  \text{Sf}_{\chi} = \frac{\bavg{w'c_\chi'}\,z}{\nu_\chi\Delta\overline{c_\chi}}
  $.
  Below, Dissipation $-$ Gradient production:
  \begin{center}
  \includegraphics[width=0.6\textwidth]{phit-phih-sf.pdf}
  \end{center}}
  
\end{minipage}
\begin{minipage}[t]{0.4\textwidth}\centering
  \vspace*{0pt}
  {\small Furnas, temperature}\\
  \includegraphics[width=\textwidth]{fu-phitt.pdf}

  {\small Verde, temperature and $\mathrm{CO_2}$}\\
  \includegraphics[width=\textwidth]{green-tt.pdf}
  \includegraphics[width=\textwidth]{green-cc.pdf}

  
\end{minipage}
  

\msection{\parbox{0.8\textwidth}{Corrections:  classification and spectral approach pushing really hard}}

%\msection{Corrections:  classification and spectral approach pushing really hard}

\mslide[Types of corrections]

\cite{foken.leuning.ea--correctionsanddata}:
\begin{enumerate}
\item Wind vector component line averaging.
\item High-cut filtering due to spatial separation.
\item High-frequency loss (sensor frequency response).
\item Tube attenuation.
{\color{SlateBlue} \item Scalar line averaging by the gas analyzer.}
\end{enumerate}
(Will not discuss the last one)


Plus: flow distortion by the sonic anemometer \citep{pena2019method}.







\mslide[Measured versus real fluxes: spectral corrections]
\pause
\[
\aavg{w'(\vut{x})c'(\vut{x})} \qquad \times \qquad
\aavg{\tilde{w}'(\vut{x})\tilde{c}'(\vut{x}?)} \qquad \times \qquad
\bavg{\tilde{w}'(\vut{x})\tilde{c}'(\vut{x}?)}
\]
Cross spectra are very useful ($\Phi$s inaccessible; $F$s accessible via
Taylor's frozen turbulence):
\begin{align*}
  R_{wc}(\vut{r}) &\equiv \aavg{w'(\vut{x})c'(\vut{x}+\vut{r})} =
  \int_{\vut{k}\in\mathbb{R}^3}
  \Phi_{wc}(\vut{k})\,\me^{+\mi(\vut{k}\bdot\vut{r})}\, \md^3\vut{k} =
  \int_{\vut{k}\in\mathbb{R}^3}
  \frac{\tilde{\Phi}_{wc}(\vut{k})}{\tau_{wc}(\vut{k},\ldots)}\,\me^{+\mi(\vut{k}\bdot\vut{r})}\,
  \md^3\vut{k},\\
  R^{1}_{wc}(r_1) &= \aavg{w'(0)c'(r_1)} =
  \int_{k_1\in\mathbb{R}} {F}_{wc}(k_1)\,\me^{+\mi k_1 r_1}\,\md{k_1} =
  \int_{k_1\in\mathbb{R}} \frac{\tilde{F}_{wc}(k_1)}{\tau^1_{wc}(k_1,\ldots)}\,\me^{+\mi k_1
    r_1}\,\md{k_1}.%\label{eq:Rij1}
\end{align*}
The transfer functions $\tau^1_{wc}$ try to correct the measured spectra
$\tilde{F}_{wc}(k_1)$ to obtain the ``true'' cross-spectrum $F_{wc}(k_1)$. Then
($\Co^{\small F}_{wc}(k_1)$ is the cospectrum),

\[
F_c = \aavg{w'(0)c'(0)} = R^1_{wc}(0) =
\int_{k_1\in\mathbb{R}}F_{wc}(k_1)\,\md{k_1} =
\int_{k_1\in\mathbb{R}}\Co^{\small F}_{wc}(k_1)\,\md{k_1}
\]

\pause Important questions:
\begin{enumerate}\tightitems
  \item Does $\Phi$ or $F$ (the ``true'') have non-zero quadrature? Maybe no.
  \item Does $\tilde{F}$ (the ``measured'') have non-zero quadrature? Often, yes.
\end{enumerate}

%% \mslide[A collection of useful facts]

%% \begin{itemize}\tightitems
%%   \item 3D Transfer functions for sonic path averaging are real
%%   \citep{kaimal.wyngaard.ea--deriving,horst.oncley--correctionstoinertial}, and
%%   do not \emph{seem} to introduce phase effects.
%%   \item In some models for $\Phi_{wc}$, it is purely real
%%   \citep{lumley--spectrum.nearly,kristensen1984effect,ogorman2003velocity-scalar},
%%   but \cite{horst.lenschow--attenuationscalarfluxes} found an indirect small
%%   quadrature effect in $F^1_{wc}$.
  
%% \end{itemize}

\msection{Line averaging and spatial separation}

\mslide[Line (or path) averaging]



\begin{minipage}[t]{0.6\textwidth}
  \vspace*{0pt}
\begin{itemize}\tightitems
\item The 3D transfer function is \emph{real} \citep{kaimal.wyngaard.ea--deriving}:
\[
\tau_{ij} = \frac{\sin^2(\vut{k}\bdot\vut{l})}{(\vut{k}\bdot\vut{l})^2}.
\]
$\Rightarrow$ no phase effects if $\phi_{ij}$ is also real ($\vut{l}$ is path,
$\vut{k}$ is wavenumber).
\item Path averaging correction is important for TKE dissipation rate estimation
\cite{freire2019effects}.
\item Path averaging also affects the measurement of sonic temperatures
\citep{horst.oncley--correctionstoinertial}
\end{itemize}
\end{minipage}
\begin{minipage}[t]{0.4\textwidth}\centering
  \vspace*{0pt}
  \includegraphics[width=0.9\textwidth]{transfercowc-t}
\end{minipage}

\textbf{For the $wc$ cospectrum and scalar fluxes}:
\begin{itemize}\tightitems
\item line averaging along $x_3$ does not affect scalar
  fluxes in unstable conditions, but may   have an effect in stable conditions
  \citep{kristensen1984effect},
\item but 3-axis anemometers have stronger attenuation
\citep{vandijk2002extension} (figure above).
\end{itemize}

\pause

\emph{Implementation of the corrections is sophisticated (geometry of the sonic paths,
numerical and analytical integration, manipulation of spectra) but worth it.}





\mslide[Spatial separation]

Important works:

\begin{itemize}
  \item \cite{lee.black--relatingeddycorrelation} (corrections based on inertial
  subrange power laws for the cospectrum),
  \item \cite{kristensen.mann.ea--howclose} (circular symmetry, small stability effects),
  \item \cite{horst.lenschow--attenuationscalarfluxes} (\ul{small quadrature effect
  in $F_{wc}^1$};  stability is important;
  best to integrate the cospectrum including lower frequencies).
\end{itemize}
\begin{align*}
  R^1_{wc}(r_1) &= \int_{k_1 = -\infty}^{+\infty} \Co^{F}_{wc}(k_1)\cos(k_1r_1)\,\md{k_1},\\
  \Co^{F}_{wc}(k_1) &= \frac{A(\mu)}{\left[1 + (kL)^{2\mu}\right]^{7/6\mu}},\\
  R^2_{wc}(r_2) &= \aavg{w'c'}\exp(-k_mr_2)
\end{align*}
and further corrections for vertical separation.


\mslide[Improvement in time]

\stepwise{\step{%
\begin{minipage}[t]{0.5\textwidth}
  \vspace*{0pt}
  \includegraphics[width=0.95\textwidth]{spatialsep-t.png}
\end{minipage}}
\step{\begin{minipage}[t]{0.5\textwidth}
Left: Figure from \cite{horst.lenschow--attenuationscalarfluxes}.
\begin{itemize}
   \item We still know very little about the lateral and vertical separation
      dependencies, \ldots
   \item because we know (very?) little about the 3-D $\Phi_{wc}$
   cross-spectrum.
\end{itemize}
\step{\emph{There is  still room for improving on spatial separation
    corrections:}
  \begin{itemize}\tightitems
    \item How do we correct if MOST does not apply?
    \item How do we correct if turbulence intensity is high and frozen
    turbulence is a poor assumption?
  \end{itemize}
}

\end{minipage}}}

\msection{Frequency response and tube attenuation}

\mslide[Frequency response and tube attenuation]

\begin{itemize}
  \item Sonic anemometers: adequate frequency response ($\ge \mathrm{20\,Hz}$).
  \item \ce{H2O}, \ce{CO2} and \ce{CH4} infrared gas analyzers: adequate
  frequency response ($\ge \mathrm{20\,Hz}$).
\end{itemize}
\pause

Are frequency response corrections a thing of the past?

\pause
No:
\begin{itemize}
  \item Tube attenuation is still an important issue.
  \item Adequate (portable; high accuracy) limited response analyzers may become
  available for more gases.
  \item Alternative limited response \ce{H2O} (\textit{e.g.} CS500) and \ce{CO2}
  (\textit{e.g.} GMP343) sensors \emph{can} be used to measure fluxes. They may
  be advantageous as backup and in rainy conditions.
\end{itemize}

\pause

\emph{Frequency response corrections are readily incorporated into Eulerian
  tower measurements by means of Taylor's frozen turbulence hypothesis.}


\mslide[Background for frequency response ]


Frequency response: the scalar sensor response time is $T_c$.
\begin{align}
  \mderiva{\tilde{c}}{t} + \frac{\tilde{c}}{T_c} &= \frac{c}{T_c} \; \Rightarrow\label{eq:auto-Tc}\\
  \Co_{wc}(n) &= \Co_{w\tilde{c}}(n) + 2\uppi n T_c \Qu_{w\tilde{c}}(n), \label{eq:Co-correct}\\
  \Qu_{wc}(n) &= -2\uppi n T_c \Co_{w\tilde{c}}(n)+ \Qu_{w\tilde{c}}(n)\label{eq:Qu-correct}.
\end{align}
The zero-phase hypothesis ($\Qu_{wc}(n)\equiv 0$) gives the \emph{widely 
used equation (in models for correction factors)}
\begin{equation}
\Co_{wc}(n) = \left( 1 + 4\uppi^2 n^2 T_c^2\right)\Co_{w\tilde{c}}(n),\label{eq:Co-badcor}
\end{equation}
With the above equation and a simplified equation for the $wc$ cospectrum,
\cite{horst--simple} obtained:
\begin{equation}
\overline{{w}'\tilde{c}'} = \frac{1}{1 + (2\uppi  f_m T_c \overline{u})^\alpha}\,\overline{w'c'}
\end{equation}
but this is not the best way to do it! Instead, once $T_c$ is known, it is better
to apply (\ref{eq:Co-correct}) directly and integrate the \emph{recovered}
cospectrum.



\pause

\vfill

Call it the RMQ method (Recovered from measured quadrature)

\vfill

\mslide[Proof of concept: with perfect knowledge of the attenuation $T_c$]

$w$ and $\theta$ measured over short grass (30-minute run)
$\theta$ low-pass filtered with (\ref{eq:auto-Tc}) and $T_c$ = 1, 10 and 100 seconds.


\includegraphics[width=0.48\textwidth]{comanyrawtimes}\hfill
\includegraphics[width=0.48\textwidth]{allwide}

\begin{itemize}\tightitems
  \item Left, Red line: just integrate the cospectrum using correction (\ref{eq:Co-badcor}).
  \item Left, All other lines: integrate the cospectrum using correction
  (\ref{eq:Co-correct}).
  \item Right, filled symbols: \cite{horst--simple} correction; open symbols:
  proposed correction.
\end{itemize}

\mslide[and with  real sensors (CS500 \textit{vs} LI7500):]

\begin{itemize}\tightitems
  \item Need to estimate $T_c$
  \item Need to intercompare the slow sensor with a fast reference sensor
  \item Need to deal with sensor noise in the high frequencies
\end{itemize}

\begin{center}
  \includegraphics[width=\textwidth]{magicvapor}
\end{center}

\mslide[Tube attenuation]

Tube attenuation: \cite{lenschow.raupach--attenuation,foken.leuning.ea--correctionsanddata}
\[
  F_{w\tilde{c}}(k_1) = \exp\left[-\frac{DL}{V}k_1^2\right]F_{wc}(k_1)
  \]

$V$ is the ``discharge velocity'', $L$ is tube length, $D$ is the effective diffusivity.  

Closed-path sensors recommended for rainy conditions and measurements at sea.  


\begin{itemize}
  \item RMQ not tested yet
  \item Again, the transfer function is purely real.
  \item But again, the \emph{measured} quadrature spectrum is probably not null
  $\Rightarrow$ \pause
  \item Again, need further research.
\end{itemize}



%% \mslide[How well are we doing the job?]


%% \begin{center}
%%   \hspace*{1cm}\includegraphics[width=0.42\textwidth]{bestiz0E.pdf}\hfill
%%   \includegraphics[width=0.42\textwidth]{residual-inst.pdf}\hspace*{1.5cm}
%% \end{center}


  %% Left: Lake evaporation with the bulk transfer method (N. L. Dias, 2019, unpublished).\\
  %% Right: Air quality modelling
  %% \citep{armani.carvalho.ea--statistical.evaluation.new} and model evaluation.\pause


\mslide[Flow distortion by the sonic itself (and flux loss?)]

\begin{itemize}\tightitems
   \item Flow distortion corrections lead to increased sensible heat fluxes:
   between 5\% and 15\% \cite{pena2019method}.
   \item Can this effect account for non-closure of the surface energy budget?
   \item Below: Energy Budget and Hydrological Budget: Ameriflux US-Ne3 and
   enclosing Wahoo river basin: correcting EC-measured LE by 14\% closes the water budget!
\end{itemize}

\includegraphics[width=0.40\textwidth]{energy-cumulated.pdf}\includegraphics[width=0.40\textwidth]{p-et-q-06804000-cumulated.pdf}

\msection{Calibration and sensor drift}

\mslide[Issues for sustained accuracy of the LI-7500]

\begin{itemize}\tightitems
  \item Infra-red analyzers need frequent calibration (difficult in long-term
  projects).
  \item Open-path sensors suffer from window contamination.
  \item For the LI-7500 family, the non-linear calibration curve between
  absorptance and gas density introduces systematic bias \citep{fratini2014eddy}.
  \item For the LI-7500, sensor heating introduces errors in fluxes
  \citep{burba2008addressing,Jarvi2009}.
  \item We think that temperature may affect both linear intercept and slope for
  the linearized response of the LI-7500 (see below)
\end{itemize}

\begin{center}
\includegraphics[width=0.65\textwidth]{ansFig1.pdf}
\end{center}

\mslide[Field corrections]

May be useful if stable slow sensors are available (which is often the case!):
put
\begin{align*}
  \bavg{x} &= \aavg{\bavg{x}} + \delta \bavg{x}, \\
    \delta \overline{r_{c}} &= \frac{1}{\left<{\overline{\rho_d}}\right>}\delta\overline{\rho_c}+
  \frac{\left<\overline{r_c}\right>}{\left<\overline{\theta}\right>}(1+ \left<\overline{r_v}\right>\mu_v)\delta\overline{\theta}
  + \frac{\left<\overline{r_c}\right>}{\left<\overline{\rho_d}\right>}\mu_v \delta \overline{\rho_v}
  -\frac{\left<\overline{r_c}\right>}{\left<\overline{p}\right>}(\left<\overline{r_v}\right>\mu_v +
  1)\delta \overline{p},\\
  \delta\overline{r_c} &= \beta_1 \delta\overline{\rho_c} +
  \beta_2\delta\overline{\theta}+\beta_3\delta\overline{\rho_v}+\beta_4\delta\overline{p}
  \qquad\Rightarrow\qquad
  {r_c'} = \beta_1 {\rho_c'} +
  \beta_2{\theta'}+\beta_3{\rho_v'}
\end{align*}
(here angle brackets are means over approx 1-month periods)

\begin{minipage}[t]{0.49\textwidth}
  \vspace*{0pt}
\begin{center}
  \includegraphics[width=0.7\textwidth]{Fig9a-t.png}
\end{center}
\end{minipage}
\begin{minipage}[t]{0.49\textwidth}
  \vspace*{0pt}

  \includegraphics[width=\textwidth,height=0.5\textheight]{Fig10.pdf}
\end{minipage}

\mslide[Then fluxes can change substantially]

Pearson Correlation ($r$) and slopes  obtained by linear regressions ($m$) of 
$F_{c,WPL}$ \textit{versus} $F_{c,g}$.

\begin{minipage}[t]{0.49\textwidth}
  \vspace*{0pt}
\begin{tabular}{cccccc}
\hline
\multirow{2}{*}{Period}  & \multicolumn{2}{c}{Daytime} & & \multicolumn{2}{c}{Nighttime}\\ \cline{2-3} \cline{5-6}
       &  $m$  & $r$  & & $m$ & $r$\\
I & 0.85 & 1.00 & & 0.95 & 1.00 \\ 
II & 0.90 & 0.98 & & 1.00 & 0.98 \\ 
III & 0.94 & 1.00 & & 0.85 & 0.97 \\ 
IV & 0.96 & 0.98 & & 0.95 & 0.98 \\ 
V & 0.99 & 1.00 & & 1.19 & 0.98 \\ 
VI & 0.58 & 0.89 & & 0.68 & 0.96 \\ 
VII & 0.93 & 1.00 & & 0.96 & 0.99 \\ 
VIII & 0.78 & 0.99 & & 0.90 & 0.94 \\ 
IX & 0.74 & 1.00 & & 0.69 & 1.00 \\ 
X & 0.86 & 0.99 & & 0.96 & 1.00 \\
\hline
\end{tabular}
\end{minipage}
\begin{minipage}[t]{0.49\textwidth}\raggedright
  \vspace*{0pt}

  Concluding:
  \begin{itemize}
    \item We need more intercomparisons throughout the experiments.
    \item We need stable field measurements to control drift and correct fluxes.
    \item We need to take into account non-linearities and environmental effects
    on the calibration curve.
  \end{itemize}
\end{minipage}


\msection{Conclusions}

\mslide[Conclusions]

\begin{itemize}
  \item Random errors are \emph{large}; integral scales (for error estimation) may not
  exist.
  \item Modeling errors are serious. Not all scalars are created equal under
  MOST.
  \item Failure of the simple picture ``Gradient Production = Dissipation'' may be
  at the heart of the problem.
  \item After all those years, we are still learning how fast-response (and
  not-so-fast as well) gas sensors and sonic anemometers respond.
  \item Flux attenuation is an important issue. Corrections for all effects
  identified so far should be applied, \emph{although they are cumbersome}.
  \item The measured quadrature holds promise to greatly improve frequency
  response corrections.
  \item \texttt{while(true) \{calibrate();  intercompare()\}}
  \item We will still profit from improved spectral models: they are at the
  heart of our understanding of how corrections should be applied.
\end{itemize}

\mslide

{\Large

Thanks

\vfill

for your attention!}

\vfill


\small
%\fancyfoot{}
\newcommand{\ape}{}
\bibliography{abbrevs,all}
\bibliographystyle{chicago}

\end{document}
