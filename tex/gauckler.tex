\documentclass[10pt,leqno]{article}
\usepackage[T1]{fontenc}
\usepackage[latin1]{inputenc}
\usepackage[french]{babel}
\usepackage{hanging}
%\usepackage[bitstream-charter]{mathdesign}

%% \usepackage{antiqua}
%% \usepackage{CormorantGaramond}


\usepackage{tgschola}
\usepackage[scaled=1.075,vvarbb,baskerville]{newtxmath}
%\usepackage[scaled=1.075,xcharter]{newtxmath}
%\usepackage[scaled=1.075,cochineal]{newtxmath}

%\usepackage{fouriernc}

%%\usepackage{TheanoDidot}

\renewcommand{\theequation}{\Roman{equation}}

\input{math.tex}

\begin{document}
\thispagestyle{empty}
\begin{center}
  (818)
\end{center}

\vspace*{0.75cm}

\begin{center}
\textbf{M�MOIRES PR�SENT�S}
\end{center}

\hangpara{2em}{1}  \textsc{HYDRODYNAMIQUE} --- {\itshape
  �tudes th�oriques et pratiques sur l'�coulement et le mouvement des
  eaux.} Note de \textbf{\textsc{M. Ph. Gauchler}}, pr�sent�e par
  M. Morin (Extrait par l'auteur.)


\begin{center}
  (Commissaires: MM. Piobert, Morin, Combes.)
\end{center}

\guillemotleft\; Le M�moire se compose de trois parties: la premi�re traite de
l'�coulement de l'eau par les orifices; la seconde du mouvement de
l'eau dans les tuyaux de conduite, la troisi�me du mouvement de l'eau
dans le canaux et dans le rivi�res. Il �tablit des formules th�oriques
pour ces divers cas, et en v�rifie l'exactitude par la comparaison des
r�sultats qu'elles fournissent avec ceux que l'�xp�rience a donn�s.

\everypar{\guillemotright\;}

\textit{Premi�re partie} -- Partant des principes g�n�raux de la
M�canique rationelle, j'�tablis l'�quation du mouvement d'une mol�cule
liquide, renferm�e dans un syst�me de vases solides. Cette equation
est
\begin{equation}
  \frac{1}{2}v^2 = \mathrm{C} + gz - \int \frac{1}{\delta}d\varphi +
  \int\frac{1}{\delta}\deriva{\varphi}{t}dt,
\end{equation}
en supposant l'axe des $z$ vertical et dirig� de haut en bas, $\delta$
la densit� du liquide, $\varphi$ la pression �prouv�e par la mol�cule,
$g$ l'intensit� de la pesanteur, $t$ le temps et $\mathrm{C}$ une
constante.

On en d�duit l'�quation du mouvement permanent
\[
\frac{1}{2}v^2 = gz - \frac{1}{\delta}\varphi;
\]
le th�or�me de Bernoulli,
\[
P_0 + \frac{v_0^2}{2g} + z_0 =
P_1 + \frac{v_1^2}{2g} + z_1 =
P_n + \frac{v_n^2}{2g} + z_n;
\]
et le th�or�me de Torricelli,
\[
v = \sqrt{2 g h}.
\]

Consid�rant ensuite le recul que subit un vase pendant l'�coulement,
j'�tablis l'impulsion que re�oit une palette d'une courbure
quelconque, par l'effet d'une lame d'eau qui y entre et en sort
tangentiellement.

La viscosit� r�sultant d'actions mutuelles des mol�cules n'a pas
d'influence sur l'�coulement d'un liquide quand le mouvement permanent est

\end{document}
