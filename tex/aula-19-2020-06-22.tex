\RequirePackage{fix-cm}
%
\documentclass{svjour3}                     % onecolumn (standard format)
%\documentclass[smallcondensed]{svjour3}     % onecolumn (ditto)
%\documentclass[smallextended]{svjour3}       % onecolumn (second format)
%\documentclass[twocolumn]{svjour3}          % twocolumn
%
\smartqed  % flush right qed marks, e.g. at end of proof
%
\usepackage{appendix}
\usepackage{amsmath}
\usepackage{graphicx}
\usepackage{lineno}
\usepackage{array}
\usepackage{color}
\usepackage{soul}
\usepackage{longtable}
\usepackage{natbib}
%\usepackage{mathptmx}
\linenumbers

%
% \usepackage{mathptmx}      % use Times fonts if available on your TeX system
%
% insert here the call for the packages your document requires
%\usepackage{latexsym}
% etc.
%
% please place your own definitions here and don't use \def but
% \newcommand{}{}
%
% Insert the name of "your journal" with
% \journalname{myjournal}

%
\newcommand*\patchAmsMathEnvironmentForLineno[1]{%
\expandafter\let\csname old#1\expandafter\endcsname\csname #1\endcsname
\expandafter\let\csname oldend#1\expandafter\endcsname\csname end#1\endcsname
\renewenvironment{#1}%
{\linenomath\csname old#1\endcsname}%
{\csname oldend#1\endcsname\endlinenomath}}% 
\newcommand*\patchBothAmsMathEnvironmentsForLineno[1]{%
\patchAmsMathEnvironmentForLineno{#1}%
\patchAmsMathEnvironmentForLineno{#1*}}%
\AtBeginDocument{%
\patchBothAmsMathEnvironmentsForLineno{equation}%
\patchBothAmsMathEnvironmentsForLineno{align}%
\patchBothAmsMathEnvironmentsForLineno{flalign}%
\patchBothAmsMathEnvironmentsForLineno{alignat}%
\patchBothAmsMathEnvironmentsForLineno{gather}%
\patchBothAmsMathEnvironmentsForLineno{multline}%
}

\begin{document}

\title{Sonic anemometer's path-averaging effect on the estimation of
  turbulent kinetic energy dissipation rates}

\titlerunning{Path-averaging effects on second-order structure function}        % if too long for running head

\author{Livia S. Freire \and  Nelson L. Dias \and Marcelo Chamecki}

%\authorrunning{Short form of author list} % if too long for running head

\institute{Livia S. Freire \at
              Graduate Program in Environmental Engineering, Federal University of Paran{\'a}, Curitiba PR, Brazil \\
              \email{liviagrion@gmail.com}           %  \\
%             \emph{Present address:} of F. Author  %  if needed
           \and
           Nelson L. Dias \at
              Department of Environmental Engineering, Federal University of
              Paran\'a, Curitiba, PR, Brazil.
              \email{nelsonluisdias@gmail.com}
           \and
           Marcelo Chamecki \at
              Department of Atmospheric and Oceanic Sciences,
              University of California, Los Angeles, CA, USA. 
              \email{chamecki@ucla.edu}
}

\date{Received: DD Month YEAR / Accepted: DD Month YEAR}
% The correct dates will be entered by the editor


\maketitle

\begin{abstract}

  The turbulent kinetic energy dissipation rate is a fundamental
  parameter in turbulent flows, but its direct measurement in the
  atmospheric surface layer is still a challenge. Indirect estimates
  are often obtained from inertial-subrange laws for turbulence
  statistics of longitudinal wind velocity. In this study, synthetic
  turbulence data are used to investigate the impact of path-averaging
  effects present in sonic anemometer data on the inertial subrange of
  the second-order structure function. Path-averaging reduces the
  value of the second-order structure function, creating a negative
  bias in the estimates of the dissipation rate. The effect is
  dependent on the path-averaging transfer function, mean wind
  velocity and path length. A simple correction for the bias on the
  basis of existing transfer functions is applied and tested with data
  measured from two different sonic anemometers. Compared to the
  spectrum, after the correction the second-order structure function
  becomes the best statistics for indirect estimation of the
  dissipation rate, due to its lower random error and insensitivity to
  aliasing effects.

\end{abstract}

%% \begin{keywords}

%%   Second-order structure function

%% \end{keywords}



\section{Introduction}

The mean dissipation rate of turbulent kinetic energy (hereafter TKE
dissipation rate or $\varepsilon$) is one of the most fundamental
parameters of turbulent flows, as it corresponds to the rate of energy
transfer across the energy cascade of a stationary flow and it can be
used to define length and time scales characteristic of the structures
of turbulence \citep{Pop2000, Dav2004}. In the atmosphere, it has been
used for different purposes such as the inertial dissipation method
for flux estimation \citep{FaiLar1986,HsiKat1997,Hen1998}, to evaluate
anisotropy \citep{War2000,ChaDia2004}, for evaluation of scalar
similarity \citep{CanDia2012}, and for construction of a reduced TKE
phase space used to characterize disturbed surface layer
flows \citep{ChaDia2018}. Furthermore, recent studies have shown that the length scale
$\ell_\varepsilon = u_*/\varepsilon$ (where $u_*$ is friction
velocity) is the correct scale for the nondimensionalization of some
statistics measured in the Atmospheric Surface Layer (ASL)
\citep{DavKro2014, PanCha2016, ChaDia2017}. However, the lack of
direct estimates of $\varepsilon$ in the atmosphere still poses a
limitation in turbulence studies. To overcome this issue, indirect
estimates are usually obtained from the inertial subrange laws derived
by Kolmogorov for the velocity spectrum and structure function of the
longitudinal wind velocity. This method is sensitive to measurement
error, in addition to the error in the empirical Kolmogorov constants
in the case of the spectrum and the second-order structure function
\citep{ChaDia2004}. Although the third-order structure function does
have an analytical constant (4/5), second-order structure functions
usually display a clearer inertial subrange \citep{AlbPar1997}, so
that they remain more popular in the atmospheric turbulence community.


In studies of the ASL, discrepancies in the TKE dissipation rate
values obtained from different inertial-subrange statistics have been
observed. For example, \citet{ChaDia2004} obtained values estimated
from second- and third-order structure functions that are, on average,
10 and 30\% lower than the estimates using the spectrum,
respectively. \citet{AlbPar1997} and \citet{ChaDia2017} obtained
estimates from the third-order structure function that are
respectively $\sim20\%$ and $\sim 30\%$ lower than the estimates from
their second-order counterparts. Such systematic discrepancies are
often justified by the errors in the empirical constants of the
second-order statistics, or to deviations from the local isotropy
hypothesis in the ASL.


When velocity fluctuations are measured with sonic anemometers, path
averaging effects can impact the spectrum and the structure functions
in different degrees. In the atmospheric surface layer, the range of
scales in the inertial subrange captured by sonic anemometer is not
large enough to avoid these effects.  Despite this issue being well
known, its impact on the estimates of TKE dissipation rates usually
are not taken into account.

In the present study, we generate synthetic data that mimic
longitudinal velocity time series with imposed TKE dissipation rate,
second-order statistics and random errors, which are used to compare
errors and biases present in $\varepsilon$ estimates from the spectrum
and the second-order structure function, before and after the
introduction of path-averaging effects. A correction for the
path-averaging bias in the second-order structure function is proposed
and then tested with actual data from two different sonic anemometers.

Despite its importance, the third-order structure function is not
discussed in the present study, as it has a less clear inertial
subrange than its second-order counterpart and it is more impacted by
anisotropy \citep{ChaDia2004}. In addition, synthetic models that
mimic third-order turbulence statistics have a large uncertainty in
the dissipation rate, contaminating error and bias analyses such as
those performed here. Therefore, we restrict our study to second-order
statistics.


\section{Theory}

The one-dimensional, longitudinal velocity spectrum $E_{11}(k_1)$ can
be defined as twice the Fourier transform of the longitudinal
correlation function $R_{11}(r_1) = \langle
u_1(x_1)u_1(x_1+r_1)\rangle$, i.e.,
\begin{equation}
  E_{11}(k_1) \equiv \frac{2}{\pi}\int_0^\infty R_{11}(r_1) \cos(k_1r_1) dr_1,
\end{equation}
 where $u_1$ is the longitudinal velocity fluctuation, $x_1$ is the
 longitudinal spatial coordinate, $r_1$ is the longitudinal spatial
 separation, $k_1$ is the longitudinal wavenumber and $\langle \,\,\,
 \rangle$ represents ensemble average. In isotropic turbulence, 
 statistics are invariant under reflections and rotations and $R_{11}$
 can then be redefined as $u^2 f(r_1)$, where $u$ is a velocity scale
 and $f(r_1)$ is a longitudinal dimensionless correlation function
 \citep[][p. 89]{Dav2004}. Note that in this case $\int_0^\infty
 E_{11}(k_1) dk_1 = u^2$.

The $n^{\text{th}}$-order, longitudinal structure function is 
\begin{equation}
  \Delta u^n \equiv \langle [u_1(x_1+r_1) - u_1(x_1)]^n \rangle.
\end{equation}
For $n = 2$, it can also be redefined as a function of $f(r_1)$ as $\Delta u^2 =
2u^2[1-f(r_1)]$, leading to a relation with the
spectrum in the form \citep[][p. 226]{Pop2000}
\begin{equation}
  \Delta u^2(r_1) = 2 \int_0^\infty E_{11} \left[1 - \cos(k_1r_1) \right] dk_1. \label{eq:specstr}
\end{equation}
This relation highlights the fact that the spectrum and the
second-order structure function are two different ways of representing
the underlying spatial structure of the velocity field.

For the special case of the inertial subrange, the spectrum and structure function laws obtained by Kolmogorov provide a suitable way to estimate $\varepsilon$ in situations where it cannot be measured directly. For $1/L \ll k_1 \ll 1/\eta$ and $\eta \ll r_1 \ll L$ (where $L$ and $\eta$ are the integral and dissipation length scales, respectively)
\begin{equation}
  E_{11}(k_1) = \alpha \varepsilon^{2/3} k_1^{-5/3}\label{eq:spec}
\end{equation}
and 
\begin{equation}
  \Delta u^2(r_1) = \beta \varepsilon^{2/3} r_1^{2/3}. \label{eq:strfun}
\end{equation}
From (\ref{eq:specstr}), in the idealized case of an infinitely large
inertial subrange, a direct relation between the constants $\alpha$
and $\beta$ can be obtained. The conditions that guarantee a good
approximation for real turbulence data were presented by
\citet{Web1964} and \citet[][p. 356]{MonYag1981}: if
$1/k_{1,\mathrm{max}} \ll r_{1,\mathrm{min}} < r_{1,\mathrm{max}} \ll
1/k_{1,\mathrm{min}}$ (where $k_{1,\mathrm{min}}, k_{1,\mathrm{max}},
r_{1,\mathrm{min}}$ and $r_{1,\mathrm{max}}$ set the boundaries of the
inertial subrange) the well known relation $\beta \approx 4\alpha$
\citep[][p. 232]{Pop2000} is valid. This is usually the case for
large Reynolds-number flows, such as those in the ASL. However, as
will be seen in Section \ref{sec:res}, path-averaging introduced by
sonic anemometers limits the practical length of the inertial
subrange, affecting this relation. In this study, instead of
correcting the constants to include path-averaging effects, we propose
a correction to recover the condition in which $\beta \approx 4\alpha$
is valid.


\section{Data and the path-averaging effect}

\subsection{Synthetic data}

To investigate the effects of path-averaging in the estimation of
$\varepsilon$ from inertial-subrange laws, we use a synthetic model
that mimics measurements of longitudinal velocity from sonic
anemometers with a known value of $\varepsilon$. Time series of a
synthetic velocity $u_1^s$ were generated using the spectral method
developed by \citet{ShiDeo1991}. The velocity time series is given by
\begin{equation}
  u_1^s(t) = \mathrm{Re}\left\{\mathrm{FFT}\left\{S(f) \Delta f \sqrt{2} \mathrm{e}^{i\phi(f)}\right\}\right\}, \label{eq:syntur}
\end{equation}
where Re\{\,\,\,\} denotes the real part of a complex number, FFT\{\,\,\,\} is
the Fast Fourier Transform, $S(f)$ is the desired spectral density of $u_1^s(t)$
as a function of frequency $f$, $\Delta f$ is the frequency increment and
$\phi(f)$ is a random variable uniformly distributed between 0 and $2\pi$, used
to obtain independent random phase angles that create the velocity fluctuations
in the time domain. Although this model does not include non-Gaussian
statistics present in real turbulence, it provides an opportunity to evaluate
path-averaging effects separately from other sources of errors, by directly
introducing these effects in the model for $S(f)$.

Equation~(\ref{eq:syntur}) provides a time series with spectral
density corresponding exactly to the chosen model for $S(f)$, except
for small errors due to numeric truncation. To mimic the random errors
inherent to real sonic anemometer data, we defined $S(f) = F_{11}(f) +
\sigma(f)y$, where $F_{11}(f) = E_{11}(k_1) 2\pi/U$ is the spectrum in
the frequency domain, $U$ is the mean longitudinal velocity,
$\sigma(f)$ is the standard deviation of real spectra and $y$ is a
random variable. In addition, assuming a quasi-normal hypothesis, the
standard deviation of the spectrum is equal to its mean value
\citep{BenPie2010,Dia2017}, which gives $S(f) = F_{11}(f)(1 + y)$ with
$y \sim (\chi^2(2)/2 - 1)$, where $\chi^2(2)$ is a chi-square
distribution with two degrees of freedom (since the spectrum
corresponds to the square of a normally-distributed random
variable). Note that the expected value and variance of $y$ are zero
and one, respectively.

Because in this study we are focused on the inertial subrange,
$F_{11}(f)$ is chosen to correspond to Kolmogorov's law (given by
Equation~(\ref{eq:spec}) multiplied by the factor $2\pi/U$) in the
entire frequency range simulated ($5.56\times 10^{-4} \le f \le
2$\,kHz). Aliasing effects present in measured data were introduced by
resampling $u_1^s(t)$ at a frequency of 20\,Hz. Path-averaging was
simulated by using
\begin{equation}
S(f) = F_{11}(f)(1 + y)H_u(fp\pi/U),
\end{equation} 
where $H_u(fp\pi/U)$ is a transfer function obtained from the ratio of
the path-averaged to the true one-dimensional, longitudinal velocity
spectrum \citep{KaiWyn1968,HorOnc2006}. For the synthetic data we test
the simplest case where a single path length in the longitudinal
direction is present, corresponding to $H_u(fp\pi/U) =
\mathrm{sinc}^2(fp\pi/U)$, with a path-length $p = 0.15$\,m.

Apart from the aforementioned random error introduced to mimic measurement
errors, this model does not present any variability in the imposed value of
$\varepsilon$. This was the crucial reason in the choice of this simple modeling
approach over other synthetic models available in the literature (e.g. the model
proposed by \citet{JunLat1994}). The final values of $u_1^s(t)$ at 20\,Hz are
equivalent to a series obtained from a sonic anemometer, where both
aliasing and path-averaging effects are present in the spectrum. The
second-order structure function obtained from this series also presents
path-averaging effects, as will be discussed in Section \ref{sec:res}.

In order to test a path-averaging correction that can be applied to
field data, the spectrum obtained from $u_1^s(t)$ is divided by
$H_u(fp\pi/U)$ (recovering the spectrum without path-averaging), a new
velocity series is obtained by performing an inverse FFT to the
corrected spectrum, and from it a corrected second-order structure
function can be estimated. Note that, although there are methods
available to reduce the aliasing effect from the spectrum
(e.g. \citet{ChaDia2004}), they should not be applied if the goal is
to correct the second-order structure function, as the aliased energy
is real and should be present in $\langle \Delta u^2\rangle$.

To make calculations simpler, results presented in Section
\ref{sec:res} were obtained with $U = 2\pi\,\mathrm{m\,s^{-1}}$ (so
wavenumber and frequencies are equivalent) and an imposed dissipation
rate equal to $1\,\mathrm{m^2s^{-3}}$. All calculations were performed
for a 30--minute time series.

\subsection{Field data \label{sec:fielddata}}

Data collected at the \textit{Rio Verde} dam in Arauc\'aria county,
state of Paran\'a, Brazil ($25^\circ31'30''$S, $49^\circ31'30''$W) are
used to test the correction proposed for the second-order structure
function. Data were measured simultaneously by a 81000 Ultrasonic
Anemometer from Young and an IRGASON gas analyzer and sonic anemometer
from Campbell Scientific, both at 20\,Hz, from September 25 to October
4 2017. Sensors were installed a few meters above the water surface,
at the border of the dam, facing the preferential mean wind
direction. Data were separated into 30-min blocks starting at midnight
local time. Post processing and data selection were performed
following \citet{ZahCho2016}. To further select data in which an
inertial subrange is likely present, a minimum mean velocity of
$1\,\mathrm{m\,s^{-1}}$ was used as a criteria. A final number of 54
and 28 blocks for the Young and Campbell anemometers, respectively,
was obtained (the different number is due to small differences in the
data obtained from each sensor).

Estimations of TKE dissipation rate from spectrum and second-order
structure function were obtained before and after the application of
path-averaging correction. As in the synthetic data, the correction
comprised four steps: (i) estimating the energy spectrum; (ii)
estimating a corrected spectrum by dividing it by the transfer
function $H_u$; (iii) performing an inverse FFT to create a time
series whose spectrum corresponds to the corrected spectrum; and (iv)
estimating a corrected second-order structure function from it. Note
that this time series created by the inverse FFT is \textit{not} a
corrected time series, but rather a time series whose spectrum is
equal to the corrected spectrum. Based on the geometry of the sensors,
the transfer functions calculated by \citet{HorOnc2006} for the Solent
R3 and CSAT3 anemometers were used for the Young and Campbell
anemometers, respectively (the values of $H_u$ are presented in
Appendix B of \citet{HorOnc2006}).


\section{Results \label{sec:res}}

\subsection{Synthetic turbulence}

\begin{figure}\centering
  \includegraphics[width=1\textwidth]{specj.pdf}
  \caption{Synthetic spectra (left panels) and second-order structure functions
    (right panels) without path-averaging (a-b), with path-averaging (c-d) and
    with path-averaging and correction (e-f). Full data (up to 2\,kHz) in
    light-grey, resampled at 20\,Hz in dark-grey, average of 1000 tests in black
    (a, c, e); path-averaging equation in blue (c); average of 1000 tests in
    black, average $\pm$ standard deviation in grey (b, d, f). In all plots, red
    line corresponds to Kolmogorov's inertial subrange law and vertical lines
    delimit the range used to estimate $\varepsilon$.\label{fig:syntur}}
\end{figure}

Figure~\ref{fig:syntur}(a) shows an example of a spectrum generated by
the synthetic model. The random error of the model, reflected in the
level of scatter around Kolmogorov's law (imposed as the mean value),
is similar to the observed in real turbulence data (see
Figure~\ref{fig:specfield} for an example). The aliasing effect when
data frequency is reduced from 2\,kHz to 20\,Hz is clearly observed in
the mean spectrum (also shown in Figure~\ref{fig:syntur}(a)). As
expected, at this stage the corresponding second-order structure
function has a well-defined inertial subrange
(Figure~\ref{fig:syntur}(b)).

The path-averaging effect introduced in the model is shown in
Figure~\ref{fig:syntur}(c), which counter-balances most of the
aliasing in the 20\,Hz--spectrum, significantly reducing the region
where the spectrum differs from Kolmogorov's law. The second-order
structure function, on the other hand, is seriously affected by
path-averaging; a consistent reduction in its value can be observed in
the region corresponding to the inertial subrange, being larger for
lower values of $r_1$ (see figure~\ref{fig:syntur}(d)). This situation
is equivalent to the estimation of the TKE dissipation rate in the ASL
from sonic anemometer data. Note that at this stage there is
over/underestimation of the levels of the spectrum/structure function
in the inertial subrange, which directly impacts the estimation of
$\varepsilon$.

The correction of the path-averaging effect in the 20\,Hz data is
effective in recovering the original result in both the spectrum and
the second-order structure function (Figures~\ref{fig:syntur}(e) and
(f), respectively). Since aliasing affects the inertial subrange of
the spectrum, this result indicates that the best approximation to the
Kolmogorov's prediction is the structure function after the
path-averaging correction. Figure~\ref{fig:synhist} summarizes the
distribution of values of $\varepsilon$ obtained from each type of
estimation, using 1000 runs of synthetic turbulence (exact values of
mean and standard deviation are presented in Table
\ref{tab:syntur}). Without path-averaging (dotted lines), the
estimates from the spectrum have a positive bias of 31\% due to
aliasing, whereas estimates from the second-order structure function
have a negative bias of less than 1\%. With path-averaging (dashed
lines), the negative bias of the structure function estimates reaches
28\%, whereas the positive bias of the spectrum is greatly
reduced. Note that in reality the balancing effect between aliasing
and path-averaging in the spectrum will vary from run to run (among
other factors, $H_u$ is a function of $U$, for example), which can
generate different biases for the spectrum estimate depending on the
data. This is another reason that makes structure function estimation
more reliable. After correction (solid lines), estimates from
the structure function have a small positive bias of less than
1\%, whereas spectrum estimates return to their original value. This
result, combined with its smaller standard deviation, makes the
corrected structure function the best approach for estimating $\varepsilon$.

\begin{figure}\centering
  \includegraphics[width=0.5\textwidth]{hist.pdf}
  \caption{Probability density functions obtained from 1000
    realizations of synthetic turbulence resampled at 20\,Hz, without
    path-averaging (dotted lines), with path-averaging (dashed lines)
    and with path-averaging and correction (solid
    lines). \label{fig:synhist}}
\end{figure}

\begin{table}
  \begin{center}
    \caption{Values of mean $\pm$ standard deviation of $\varepsilon/\varepsilon_{\text{imposed}}$ estimated from synthetic data.}
    \begin{tabular}{lcc}
      \hline\noalign{\smallskip}
      & spectrum & 2$^{\text{nd}}$ str. fun. \\ 
      \noalign{\smallskip}\hline\noalign{\smallskip}
      without path-averaging & $1.31\pm 0.024$  & $0.99\pm 0.012$ \\
      with path-averaging & $1.05\pm 0.019$  & $0.72\pm 0.009$ \\
      corrected path-averaging & $1.32\pm 0.023$  & $1.01 \pm 0.012$ \\
      \noalign{\smallskip}\hline
    \end{tabular}
    \label{tab:syntur}
  \end{center}
\end{table}


It is important to point out that the exact values of bias are a
function of the $f$ and $r_1$ intervals used to estimate $\varepsilon$
from each function: in the present exercise, we chose typical values
used in the literature. Also, errors in spectrum estimates can be
reduced by averaging over a longer inertial subrange interval; we used
a one-decade range which is similar to the range available in
experimental data. The error in the estimation using the second-order
structure function is also dependent on the $r_1$ interval used, being
lower for smaller values (see Figure \ref{fig:syntur}). Note that
because the random errors introduced in the synthetic model are
similar to the ones present in real spectrum, these values of standard
deviation are probably a good estimate of the random errors present in
real $\varepsilon$ estimates from sonic anemometer data. Finally, the
effects of path-averaging on $\varepsilon$ estimates are smaller the
lower the path length of the sensor, as expected (for example, the
second-order structure bias without correction is reduced to $-14$\%
for $p = 0.05$m; not shown). Despite these variations in exact values
of errors and biases, the overall conclusion that the smaller error
and bias is obtained from the corrected second-order structure
function remains solid.

\subsection{Field data}

Now we test the effect of the path-averaging correction on real
turbulence data from two different sonic anemometers. Because there is
no measurement of the real dissipation rate, there is no way to know
exactly how much the error and bias are in each estimate of
$\varepsilon$. However, based on the results obtained from the
synthetic data, we expect that the estimate from the corrected
second-order structure function is the most reliable one.

Figure \ref{fig:transfer} presents examples of transfer functions
calculated by \citet{HorOnc2006} for each anemometer. Note that, while
the measurement frequency is fixed (it goes from $5.56\times 10^{-4}$
to 20\,Hz), the transfer function is impacted by the mean longitudinal
velocity. Therefore, each 30--min run will have a different
path-averaging effect.

\begin{figure}\centering
  \includegraphics[width=1\textwidth]{transfer.pdf}
  \caption{Transfer functions ($H_u$) for (a) Young and (b) Campbell
    sonic anemometers. Black solid lines with dots correspond to the
    values estimated by \citet{HorOnc2006}, as a function of the
    wavenumber normalized by the path length ($\kappa p$). Dashed
    lines correspond to $H_u$ as a function of the frequency ($f$),
    for different values of mean wind velocity ($U$). Vertical line
    corresponds to the Nyquist frequency of the measurements presented
    here. \label{fig:transfer}}
\end{figure}

Figure \ref{fig:specfield} presents an example of measured spectrum
and second-order structure function, for the run obtained on September
25 2017 at 22:30 local time by each anemometer, before and after the
path-averaging correction. Note that, as found with the synthetic
data, the correction increases the overestimation of spectral
densities due to aliasing, and it fixes the inertial subrange of the
second-order structure function. For the Young anemometer, the final
result is very close to the Kolmogorov law; for the Campbell
anemometer, there is some underestimation left, possibly due to other
sensor effects (see the average of all runs in Figure
\ref{fig:strf}).

Dissipation rates estimated from spectrum and second-order structure
function before and after correction for all runs are presented in
Figure \ref{fig:stat}. On average, dissipation increased by $\sim30\%$
in both estimates for the Young data, and by $\sim20\%$ in both
estimates for the Campbell data. Considering that the corrected
second-order structure function provides the best estimate, the
uncorrected spectrum has an overestimation of $\sim 20$--$30\%$ and the
corrected spectrum has an overestimamtion of $\sim 60\%$, on average.

\begin{figure}\centering
  \includegraphics[width=1\textwidth]{spec-verde.pdf}
  \caption{Longitudinal spectrum (left panels) and second-order
    structure function (right panels) measured by the Young anemometer
    (upper panels) and Campbell anemometers (lower panels). Light-grey
    is original spectrum, black and blue lines are smoothed spectra
    (moving average with a 100-point window). For both statistics, the
    blue line is before the path-averaging correction, and the black
    line is after. The orange line is the transfer function used. The
    red line is the slope of Kolmogorov's law. Vertical lines
    delimit the interval used for $\varepsilon$
    estimation. Measurements were made on September 25 2017 at 22:30
    local time. \label{fig:specfield}}
\end{figure}

\begin{figure}\centering
  \includegraphics[width=1\textwidth]{strf.pdf}
  \caption{Mean values (solid lines) $\pm$ one standard deviation (shaded area)
    of the normalized second-order structure function before (blue) and after
    (red) de path-averaging correction, for the Young (a) and Campbell (b)
    anemometers. Horizontal line is a fit for the mean inertial subrange of the
    Young anemometer. Values normalized by the dissipation length scale
    $\ell_\varepsilon = u_*^3/\varepsilon$ and the friction velocity $u_*$
    (values of $\varepsilon$ estimated after correction were
    used).  \label{fig:strf}}
\end{figure}

\begin{figure}\centering
  \includegraphics[width=1\textwidth]{stat2.pdf}
  \caption{(a) Comparison between dissipation rates estimated from
    spectrum (blue) and second-order structure function (red) before
    and after correction, using Young (dots) and Campbell (squares)
    anemometers. On average, spectrum estimates increased by 31\%
    (Young) and 23\% (Campbell) after correction, whereas second-order
    structure function estimates increased 33\% (Young) and 23\%
    (Campbell). (b) Dissipation rates estimated from uncorrected (red)
    and corrected (blue) spectrum, compared with corrected
    second-order structure function estimates, using Young (dots) and
    Campbell (squares) anemometers. On average, spectrum estimates
    are overestimated by 23\% (Young) and 30\% (Campbell) when
    uncorrected, and by 63\% (Young) and 60\% (Campbell) when
    corrected.\label{fig:stat}}
\end{figure}


\section{A note on the CSAT3 sonic anemometer by Campbell Scientific}

In the field experiment described in Section \ref{sec:fielddata}, data
from a CSAT3 sonic anemometer by Campbell Scientific were also
collected; therefore, a similar correction as performed in the IRGASON
data could also be tested. However, the second-order structure
function from CSAT3 sonics presents a distinct feature that cannot be
explained by the path-averaging effect, or any other effect known to
the authors. In the lowest values of the time lag $\tau$, within the
inertial subrange, there is an increase in values of $\langle \Delta
u^2\rangle$ compared to Kolmogorov's law, instead of the decrease
observed due to path-averaging (Figure \ref{fig:csat}). This feature
seems to be characteristic of the CSAT3 model (and not of a specific
sensor), as a similar behavior can be observed in data measured by
different sensors in different field experiments (see for example data
from the AHATS \citep{SalCha2012} and the GoAmazon \citep{FueCha2016}
experiments). None of the known sonic anemometer effects discussed in
the literature (e.g. flow distortion by transducer shadowing,
\citep{HorSem2015}) could be the cause of this behavior; an
investigation is left for future studies.

\begin{figure}\centering
  \includegraphics[width=1\textwidth]{spec-csat.pdf}
  \caption{Compensated longitudinal (a) spectrum and (b) second-order
    structure function measured by CSAT3 sonic anemometer from
    \textit{Rio Verde} dam experiment (September 25 2017 22:30 local
    time, black), AHATS experiment (June 26 2008 03:30 local time
    measured at 1.51\,m above ground, blue) and GoAmazon experiment
    (March 29 2014 00:00 local time measured at $z/h = 1.38$, where $h
    = 35$\,m is canopy height, grey). The corresponding IRGASON data
    from \textit{Rio Verde} dam experiment is presented in
    orange. Dashed red lines correspond to the slope of Kolmogorov's
    law. No correction to the data is applied.\label{fig:csat}}
\end{figure}

\section{Conclusion}

Let's make a deal: \cite{LETSMAKEADEAL--INFO}.

Turbulence data measured by sonic anemometers present a discrepancy
between the structure function and spectrum estimates of TKE
dissipation rate using Kolmogorov's theory for the inertial
subrange. While aliasing creates a positive bias in estimates from the
spectrum depending on how close to the Nyquist frequency the
estimation is made, it does not affect structure function
estimates. Path-averaging effects, on the other hand, counter-balance
part of the aliasing in spectrum estimates, while creating a negative
bias in the value of $\varepsilon$ estimated from the second-order
structure function. The ubiquitous presence of both effects in typical
atmospheric turbulence data explains the difference in those estimates
observed in the literature.

Path-averaging effects, quantified here using synthetic
data, can be observed in real turbulence data measured with two different sonic
anemometers. Using transfer functions for the spectrum based on the anemometer
geometry, it is possible to remove the path-averaging
effect from the data, and reconstruct the inertial subrange in the
second-order structure function. This correction increased the estimate of
$\varepsilon$, and it was successful in recovering the inertial subrange
from the 81000 Ultrasonic Anemometer by Young, whereas the recovery of data from
the IRGASON gas analyzer and sonic anemometer by Campbell Scientific was
partial. Although a similar correction could be applied to data from the
CSAT3 sonic
anemometer by Campbell Scientific, in this case the second-order structure
function presented an inertial subrange behavior that does not correspond to
path-averaging, and for this reason the correction could not be tested.

Finally, the use of the corrected second-order structure function to
estimate TKE dissipation rate is likely the best approach, as it is
unbiased and presents smaller random error. This is valid for
different types of sonic anemometers, as long as a transfer function
derived from its geometry is available.

\begin{acknowledgements}
  LSF was funded by the Brazilian National Council for Scientific and
  Technological Development (CNPq) under research grant 401019/2017-9;
  NLD was also funded by CNPq under research grants 444462/2014-7 and
  303581/2013-1. We thank Fernando Armani, Rodrigo Branco Rodakoviski,
  Lucas Emilio B. Hoeltgebaum, Andr\'e Lu\'is Diniz dos Santos, Bianca
  Luhm Crivellaro and Dornelles Vissotto Junior for data collection
  and post processing of the \textit{Rio Verde} dam experiment, and
  Andr\'e Luiz Grion for insightful discussions.
\end{acknowledgements}

% BibTeX users please use one of
%\bibliographystyle{spbasic}      % basic style, author-year citations
%\bibliographystyle{spmpsci}     % mathematics and physical sciences
%\bibliographystyle{spphys}      % APS-like style for physics
%\bibliography{pathavg}           % name your BibTeX data base

\begin{thebibliography}{26}
\providecommand{\natexlab}[1]{#1}
\providecommand{\url}[1]{{#1}}
\providecommand{\urlprefix}{URL }
\expandafter\ifx\csname urlstyle\endcsname\relax
  \providecommand{\doi}[1]{DOI~\discretionary{}{}{}#1}\else
  \providecommand{\doi}{DOI~\discretionary{}{}{}\begingroup
  \urlstyle{rm}\Url}\fi
\providecommand{\eprint}[2][]{\url{#2}}

\bibitem[{Albertson et~al.(1997)Albertson, Parlange, Kiely, and
  Eichinger}]{AlbPar1997}
Albertson JD, Parlange MB, Kiely G, Eichinger WE (1997) The average dissipation
  rate of turbulent kinetic energy in the neutral and unstable atmospheric
  surface layer. Journal of Geophysical Research: Atmospheres
  102(D12):13423--13432, \doi{10.1029/96JD03346},
  \urlprefix\url{http://dx.doi.org/10.1029/96JD03346}

\bibitem[{Bendat and Piersol(2010)}]{BenPie2010}
Bendat JS, Piersol AG (2010) Random Data, 4th edn. John Wiley \& Sons

\bibitem[{Cancelli et~al.(2012)Cancelli, Dias, and Chamecki}]{CanDia2012}
Cancelli DM, Dias NL, Chamecki M (2012) Dimensionless criteria for the
  production-dissipation equilibrium of scalar fluctuations and their
  implications for scalar similarity. Water Resources Research 48(10):n/a--n/a,
  \doi{10.1029/2012WR012127},
  \urlprefix\url{http://dx.doi.org/10.1029/2012WR012127}

\bibitem[{Chamecki and Dias(2004)}]{ChaDia2004}
Chamecki M, Dias NL (2004) The local isotropy hypothesis and the turbulent
  kinetic energy dissipation rate in the atmospheric surface layer. Quarterly
  Journal of the Royal Meteorological Society 130(603):2733--2752,
  \doi{10.1256/qj.03.155}, \urlprefix\url{http://dx.doi.org/10.1256/qj.03.155}

\bibitem[{Chamecki et~al.(2017)Chamecki, Dias, Salesky, and Pan}]{ChaDia2017}
Chamecki M, Dias NL, Salesky ST, Pan Y (2017) Scaling laws for the longitudinal
  structure function in the atmospheric surface layer. Journal of the
  Atmospheric Sciences 74(4):1127--1147, \doi{10.1175/JAS-D-16-0228.1},
  \urlprefix\url{https://doi.org/10.1175/JAS-D-16-0228.1}

\bibitem[{Chamecki et~al.(2018)Chamecki, Dias, and Freire}]{ChaDia2018}
Chamecki M, Dias NL, Freire LS (2018) A {TKE}-based framework for studying
  disturbed atmospheric surface layer flows and application to vertical
  velocity variance over canopies. Geophysical Research Letters
  45(13):6734--6740, \doi{10.1029/2018GL077853},
  \urlprefix\url{https://doi.org/10.1029/2018GL077853}

\bibitem[{Davidson(2004)}]{Dav2004}
Davidson P (2004) Turbulence, an Introduction for scientists and engineerers.
  Oxford University Press

\bibitem[{Davidson and Krogstad(2014)}]{DavKro2014}
Davidson P, Krogstad P{\r A} (2014) A universal scaling for low-order structure
  functions in the log-law region of smoothand rough-wall boundary layers.
  Journal of Fluid Mechanics 752:140--156, \doi{10.1017/jfm.2014.286},
  \urlprefix\url{http://dx.doi.org/10.1017/jfm.2014.286}

\bibitem[{Dias(2017)}]{Dia2017}
Dias NL (2017) Smoothed spectra, ogives, and error estimates for atmospheric
  turbulence data. Boundary-Layer Meteorology 166(1):1--29,
  \doi{10.1007/s10546-017-0293-7},
  \urlprefix\url{https://doi.org/10.1007/s10546-017-0293-7}

\bibitem[{Fairall and Larsen(1986)}]{FaiLar1986}
Fairall CW, Larsen SE (1986) Inertial-dissipation methods and turbulent fluxes
  at the air-ocean interface. Boundary-Layer Meteorology 34(3):287--301,
  \doi{10.1007/BF00122383}, \urlprefix\url{https://doi.org/10.1007/BF00122383}

\bibitem[{Fuentes et~al.(2016)Fuentes, Chamecki, dos Santos, Randow, Stoy,
  Katul, Fitzjarrald, Manzi, Gerken, Trowbridge, Freire, Ruiz-Plancarte, Maia,
  T\'{o}ta, Dias, Fisch, Schumacher, Acevedo, and Mercer}]{FueCha2016}
Fuentes JD, Chamecki M, dos Santos RMN, Randow CV, Stoy PC, Katul G,
  Fitzjarrald D, Manzi A, Gerken T, Trowbridge A, Freire LS, Ruiz-Plancarte J,
  Maia JMF, T\'{o}ta J, Dias N, Fisch G, Schumacher C, Acevedo O, Mercer JR
  (2016) Linking meteorology, turbulence, and air chemistry in the amazon
  rainforest. Bulletin of the American Meteorological Society
  97(12):2329--2342, \doi{10.1175/BAMS-D-15-00152.1},
  \urlprefix\url{http://dx.doi.org/10.1175/BAMS-D-15-00152.1}

\bibitem[{Henjes(1998)}]{Hen1998}
Henjes K (1998) Justification of the inertial dissipation technique in
  anisotropic mean flow. Boundary-Layer Meteorology 88(2):161--180,
  \doi{10.1023/A:1001138218181},
  \urlprefix\url{https://doi.org/10.1023/A:1001138218181}

\bibitem[{Horst and Oncley(2006)}]{HorOnc2006}
Horst TW, Oncley SP (2006) Corrections to inertial-range power spectra measured
  by csat3 and solent sonic anemometers,1. path-averaging errors.
  Boundary-Layer Meteorology 119(2):375--395, \doi{10.1007/s10546-005-9015-7},
  \urlprefix\url{https://doi.org/10.1007/s10546-005-9015-7}

\bibitem[{Horst et~al.(2015)Horst, Semmer, and Maclean}]{HorSem2015}
Horst TW, Semmer SR, Maclean G (2015) Correction of a non-orthogonal,
  three-component sonic anemometer for flow distortion by transducer shadowing.
  Boundary-Layer Meteorology 155(3):371--395, \doi{10.1007/s10546-015-0010-3},
  \urlprefix\url{http://dx.doi.org/10.1007/s10546-015-0010-3}

\bibitem[{Hsieh and Katul(1997)}]{HsiKat1997}
Hsieh CI, Katul GG (1997) Dissipation methods, taylor's hypothesis, and
  stability correction functions in the atmospheric surface layer. Journal of
  Geophysical Research: Atmospheres 102(D14):16391--16405,
  \doi{10.1029/97JD00200},
  \urlprefix\url{https://agupubs.onlinelibrary.wiley.com/doi/abs/10.1029/
  97JD00200}

\bibitem[{Juneja et~al.(1994)Juneja, Lathrop, Sreenivasan, and
  Stolovitzky}]{JunLat1994}
Juneja A, Lathrop DP, Sreenivasan KR, Stolovitzky G (1994) Synthetic
  turbulence. Phys Rev E 49:5179--5194, \doi{10.1103/PhysRevE.49.5179},
  \urlprefix\url{https://link.aps.org/doi/10.1103/PhysRevE.49.5179}

\bibitem[{Kaimal et~al.(1968)Kaimal, Wyngaard, and Haugen}]{KaiWyn1968}
Kaimal JC, Wyngaard JC, Haugen DA (1968) Deriving power spectra from a
  three-component sonic anemometer. Journal of Applied Meteorology
  7(5):827--837, \doi{10.1175/1520-0450(1968)007<0827:DPSFAT>2.0.CO;2},
  \urlprefix\url{https://doi.org/10.1175/1520-0450(1968)007<0827:
  DPSFAT>2.0.CO;2}

\bibitem[{Monin and Yaglom(1981)}]{MonYag1981}
Monin A, Yaglom A (1981) Statistical Fluid Mechanics, Volume II: Mechanics of
  Turbulence, 2nd edn. The MIT Press,
  \urlprefix\url{https://books.google.com.br/books?id=6xPEAgAAQBAJ}

\bibitem[{Pan and Chamecki(2016)}]{PanCha2016}
Pan Y, Chamecki M (2016) A scaling law for the shear-production range of
  second-order structure functions. Journal of Fluid Mechanics 801:459--474,
  \doi{10.1017/jfm.2016.427},
  \urlprefix\url{http://dx.doi.org/10.1017/jfm.2016.427}

\bibitem[{Pope(2000)}]{Pop2000}
Pope SB (2000) Turbulent Flows. Cambridge University Press

\bibitem[{Salesky and Chamecki(2012)}]{SalCha2012}
Salesky ST, Chamecki M (2012) Random errors in turbulence measurements in the
  atmospheric surface layer: Implications for monin-obukhov similarity theory.
  Journal of the Atmospheric Sciences 69(12):3700--3714,
  \doi{10.1175/JAS-D-12-096.1},
  \urlprefix\url{http://dx.doi.org/10.1175/JAS-D-12-096.1}

\bibitem[{Shinozuka and Deodatis(1991)}]{ShiDeo1991}
Shinozuka M, Deodatis G (1991) Simulation of stochastic processes by spectral
  representation. Applied Mechanics Reviews 44(4):191--204,
  \doi{10.1115/1.3119501}, \urlprefix\url{http://dx.doi.org/10.1115/1.3119501}

\bibitem[{{The Official Let's Make a Deal Website}(2020)}]{LETSMAKEADEAL--INFO}
{The Official Let's Make a Deal Website} (2020) Show info. Obtido em
  \url{http://www.letsmakeadeal.com/showinfo.htm}

\bibitem[{Warhaft(2000)}]{War2000}
Warhaft Z (2000) Passive scalars in turbulent flows. Annual Review of Fluid
  Mechanics 32(1):203--240, \doi{10.1146/annurev.fluid.32.1.203},
  \urlprefix\url{https://doi.org/10.1146/annurev.fluid.32.1.203}

\bibitem[{Webb(1964)}]{Web1964}
Webb EK (1964) Ratio of spectrum and structure-function constants in the
  inertial subrange. Quarterly Journal of the Royal Meteorological Society
  90(385):344--346, \doi{10.1002/qj.49709038520},
  \urlprefix\url{http://dx.doi.org/10.1002/qj.49709038520}

\bibitem[{Zahn et~al.(2016)Zahn, Chor, and Dias}]{ZahCho2016}
Zahn E, Chor TL, Dias NL (2016) A simple methodology for quality control of
  micrometeorological datasets. American Journal of Environmental Engineering
  6(4A):135--142, \doi{10.5923/s.ajee.201601.20},
  \urlprefix\url{https://doi.org/10.5923/s.ajee.201601.20}

\end{thebibliography}



\end{document}
% end of file template.tex

